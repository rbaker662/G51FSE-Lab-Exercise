\section{First Member}
This is the section dedicated to one of the team members, and it should be written individually . It can include a range of things; first subsection is a space for you to point out the strengths and weaknesses of the module, including complaints about the module coordinator Max Wilson. The second section should have a selfie image with Max! The last part of it is the most important one. You will need to write a paragraph about what you have learned in this module. You can write it in \textbf{Bold} if you want or you can use other fonts. 

Please do not forget:
\begin{itemize}
	\item First paragraph should have your comments about the module
	\item Second one, a selfie img with Max
	\item Last one, what you learned in this module.
\end{itemize}

\subsection{Comments about the module}
This is the first subsection. Do not forget to include the best parts about the module as well as what you did not like about Max Wilson during the term.
The module was OK it meant working in a team was good the worst part was that my team was a member short which meant in the timed lab we often ran out of time.
\subsection{Selfie with Max}

To include an image, you will need to remove the comments from the code below, place an image in the main folder, and do not forget to put the name of the image instead of ImgName. 

begin{figure}
caption{Photo of rubik's cube}
centering
includegraphics[width=0.5]{Image.png}
label{fig:selfie}
end{figure}

You can then use the label of the figure to reference it later with the command {\backslash}ref. you can comment out the next line to see an example of how it works.

My selfie with Max is in  Figure~\ref{fig:selfie}.

\subsection{What I have learned in this module}
In this project I have learned that you have to rely on people. The planning and prototyping and testing are just as important as the coding to avoid problems after release.

